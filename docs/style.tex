\usepackage{geometry}
\usepackage{fancyhdr}
\usepackage{makecell}
\usepackage{multirow}
\usepackage{listings}
\usepackage{tikz}
\usepackage{fontspec, xunicode, xltxtra}
\usepackage{titlesec}
\usepackage{float}
\usepackage{caption}
\usepackage{amsmath, amssymb, amsfonts}
\usepackage[colorlinks, linkcolor=black, anchorcolor=black, citecolor=black]{hyperref}
\usepackage[justification=centering]{caption}
\usepackage{subcaption}
\usepackage{color}
\usepackage[super,square]{natbib}
\usepackage{blindtext}
\usepackage{bm}
\usepackage{enumitem}
\usepackage{graphicx}
\usepackage{wrapfig}
\usepackage{algorithm2e}
\usepackage[simplified]{pgf-umlcd}
\usepackage[lining]{ebgaramond}
\usepackage{lmodern}
\usepackage{longtable, booktabs}

% use print-styles when defining "PrintStyle"
\if @twoside % test onesize or twoside
  \def \PrintStyle {}
\fi

% set the margin
\geometry{left=3.18cm, right=3.18cm, top=2.54cm, bottom=2.54cm}

% title and context interval
\titlespacing*{\chapter} {0pt}{0pt}{20pt}
\titlespacing*{\section}{0pt}{3ex}{1ex}
\titlespacing*{\subsection}{0pt}{2ex}{1ex}
\titlespacing*{\subsubsection}{0pt}{1ex}{0.5ex}
\titlespacing{\paragraph}{0pt}{1.5ex minus .1ex}{1pc}

\titleformat{\section}{\Large\bfseries}{\thesection}{0.75em}{}
\titleformat{\subsection}{\large\bfseries}{\thesubsection}{0.75em}{}
\titleformat{\paragraph}[runin]{\normalsize\sffamily}{\theparagraph}{}{}

% list line-spacing
\setenumerate[1]{itemsep=1.5pt,partopsep=0pt,parsep=\parskip,topsep=5pt}
\setitemize[1]{itemsep=1.5pt,partopsep=0pt,parsep=\parskip,topsep=5pt}
\setdescription{itemsep=1.5pt,partopsep=0pt,parsep=\parskip,topsep=5pt}

% set page number font
\fancyfoot[C]{\mdseries\small\thepage}

\fancypagestyle{plain}{%
    \fancyhf{} % clear all header and footer fields
    \fancyfoot[C]{\mdseries\small{\thepage}} % except the center
    \renewcommand{\headrulewidth}{0pt}
    \renewcommand{\footrulewidth}{0pt}
}

\newcommand{\chapterformat}{\centering\huge\bf\thispagestyle{plain}}

% set the tikz
\usetikzlibrary{chains,fit,shapes}

% set the code style
\ifx \PrintStyle \undefined
    \def \highlightcolor {\color[rgb]{0.5,0,0.5}}
    \lstset{
        commentstyle=\small\color{green!50!black}\itshape,
        keywordstyle=\small\color{cyan!50!black}\bfseries,
        stringstyle=\small\color{purple},
        numberstyle=\small\ttfamily\color{gray},
    }
\else   % print style
    \def \highlightcolor {}
    \lstset{
        commentstyle=\small\itshape,
        keywordstyle=\small\bfseries,
        numberstyle=\small\ttfamily,
    }
\fi

% generic code style
\lstset{
    basicstyle=\small\ttfamily,
    columns=fullflexible,
    numbersep=1em,
    xleftmargin=\parindent,
    breakatwhitespace=false,
    numbers=left,
    captionpos=b,               % sets the caption-position to bottom
    breaklines=true,            % automatic line breaking only at whitespace
    keepspaces=true,
    showstringspaces=false,
    tabsize=4
}

% define the language berry
\lstdefinelanguage{berry} {%
    keywords={def,var,if,elif,else,while,for,end,
              break,return,continue,true,false,nil,
              do,import,class,as,try,except,raise},%
    morekeywords={print,type,self,super},
    morecomment=[l]\#,%
    morecomment=[s]{\#-}{-\#},%
    morestring=[b]",%
    morestring=[b]',%
}

% define the language ebnf
\lstdefinelanguage{ebnf} {%
    keywords={ID,INTEGER,REAL,STRING},%
    morecomment=[s]{(*}{*)},%
    morestring=[b]",%
    morestring=[b]',%
}

\lstdefinestyle{berry} {%
    morekeywords={bint,breal,bvm,size_t}%
}

% define the library function title style
\newcommand{\libtitle}[1]{\subsubsection{\highlightcolor\textsl{#1}}}
% define the FFI title style
\newcommand{\ffititle}[1]{\subsubsection{\texttt{\highlightcolor\textsl{#1}}}}
